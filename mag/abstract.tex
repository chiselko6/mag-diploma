\sectioncentered*{Реферат}
% \thispagestyle{empty}
\setcounter{page}{2}

% Зачем: чтобы можно было вывести общее число страниц.
% Добавляется единица, поскольку последняя страница -- ведомость.
\FPeval{\totalpages}{round(\getpagerefnumber{LastPage} + 1, 0)}

\begin{center}
	Пояснительная записка \num{\totalpages}~с., \num{\totfig{}}~рис., \num{\tottab{}}~табл., \num{\toteq{}}~формул, \num{\totref{}}~источников.
	\MakeUppercase{Программное средство, веб-приложение, университет, организация учебного процесса, расписание, индивидуальные задания}
\end{center}

Объектом исследования данного дипломного проекта является создание приложения на платформе \LB для сферы розничных продаж.

Цель работы – разработка программного продукта, автоматизирующий и оптимизирующий операции с данными в сфере розничной торговли, его оптимизация для увеличения быстродействия.

Использование данного программного продукта позволит упростить управление движением товаров торговой сети. Уменьшается количество т.н. «бумажной» работы, поскольку все операции автоматизированы, уменьшается вероятность ошибки благодаря протестированным системам подсчета. Наглядное представление количества проданных товаров, а также модуль прогноза продаж помогает видеть картину в деталях.

В рамках работы над проектом изучена структура и принцип работы платформы \LB, выявлены ее характерные особенности, изучены подходы к построению 3-уровневых архитектур, механизмы работы баз данных.

В ходе разработки программного продукта было проведено тестирование данных, используемых в качестве тестовых. Кроме того, была протестирована надежность данного приложения.

В разделе технико-экономического обоснования был произведен расчёт затрат на создание ПО, а также прибыли от разработки, получаемой разработчиком. Проведенные расчёты показали экономическую целесообразность проекта.
