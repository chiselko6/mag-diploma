\sectioncentered*{Общая характеристика работы}
\addcontentsline{toc}{section}{Общая характеристика работы}

\textbf{Цель и задачи исследования}

\emph{Целью} диссертационной работы является  оптимизация программного продукта на платформе \LB, автоматизирующего операции с данными в сфере розничной торговли, для увеличения производительности обработки сложных запросов и быстродействия при пересчете планов.

\emph{Объектом} исследования является готовый программный продукт, используемый крупными торговыми компаниями.

\emph{Предметом} исследования является исходный код продукта, платформа и база данных \LB.\\


% \textbf{Связь работы с приоритетными направлениями научных исследова- ний и запросами реального сектора экономики}

\textbf{Личный вклад соискателя}

Результаты, приведенные в диссертации, получены соискателем лично при изучении и анализе работы платформы \LB.\\

% \textbf{Апробация результатов диссертации}

\textbf{Опубликованность результатов диссертации}

Не публиковалась.\\

\textbf{Структура и объем диссертации}

% \thispagestyle{empty}
Диссертация состоит из введения, общей характеристики работы, трех глав, заключения, списка использованных источников, списка публикаций автора. В первой главе произведен обзор предметной области - какие требования ставятся при построении приложений для розничной торговли, как разрабатываются такие приложения в общих случаях и какие частные практики используются при построении на платформе \LB. Вторая глава посвящена используемым технологиям, в частности дается разбор составляющих платформы, используемого языка запросов, веб-сервисов. Третья глава содержит непосредственно детальный разбор проблем при работе с базой данных и их решений. В заключении подводится краткий итог всех примененных оптимизаций.

% Зачем: чтобы можно было вывести общее число страниц.
% Добавляется единица, поскольку последняя страница -- ведомость.
\FPeval{\totalpages}{round(\getpagerefnumber{LastPage} + 1, 0)}

% bad figure counter !!!!
Общий объем работы составляет \num{\totalpages}~с., \num{21}~рис., \num{\tottab{}}~табл., \num{\toteq{}}~формул, \num{\totref{}}~источников.

% \begin{center}
% 	Пояснительная записка \num{\totalpages}~с., \num{\totfig{}}~рис., \num{\tottab{}}~табл., \num{\toteq{}}~формул, \num{\totref{}}~источников.
% 	\MakeUppercase{Программное средство, веб-приложение, университет, организация учебного процесса, расписание, индивидуальные задания}
% \end{center}

% Объектом исследования данного дипломного проекта является создание приложения на платформе \LB для сферы розничных продаж.
%
% Цель работы – разработка программного продукта, автоматизирующий и оптимизирующий операции с данными в сфере розничной торговли, его оптимизация для увеличения быстродействия.
%
% Использование данного программного продукта позволит упростить управление движением товаров торговой сети. Уменьшается количество т.н. "бумажной" работы, поскольку все операции автоматизированы, уменьшается вероятность ошибки благодаря протестированным системам подсчета. Наглядное представление количества проданных товаров, а также модуль прогноза продаж помогает видеть картину в деталях.
%
% В рамках работы над проектом изучена структура и принцип работы платформы \LB, выявлены ее характерные особенности, изучены подходы к построению 3-уровневых архитектур, механизмы работы баз данных.
%
% В ходе разработки программного продукта было проведено тестирование данных, используемых в качестве тестовых. Кроме того, была протестирована надежность данного приложения.
%
% В разделе технико-экономического обоснования был произведен расчёт затрат на создание ПО, а также прибыли от разработки, получаемой разработчиком. Проведенные расчёты показали экономическую целесообразность проекта.
