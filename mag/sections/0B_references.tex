% Зачем: Изменение надписи для списка литературы
% Почему: Пункт 2.8.1 Требований по оформлению пояснительной записки.
\renewcommand{\bibsection}{\sectioncentered*{Библиографический список}\\
\textbf{\large Список использованных источников}}
\addcontentsline{toc}{section}{Библиографический список}
\addcontentsline{toc}{subsection}{Список использованных источников}
\phantomsection\pagebreak % исправляет нумерацию в документе и исправляет гиперссылки в pdf

% Зачем: включить в список литературы все источники из базы (даже если на них нет ссылок в тексте).
% \nocite{*}

% Зачем: Печать списка литературы. База данных литературы - файл bibliography_database.bib
\bibliography{bibliography_database}
\hfill

\textbf{\large Список публикаций соискателя}
\addcontentsline{toc}{subsection}{Список публикаций соискателя}\\

1-A. Тишковский М.А., Лимонтов А.С., Евжик Д.А. Эффективное встроенное ансамблирование нейронных сетей / Тишковский М.А., Лимонтов А.С., Евжик Д.А. // 55-я юбилейная научная конференция аспирантов, магистрантов и студентов. – 2019. – с. 220-221.

2-А. Тишковский М.А., Лимонтов А.С., Подвальников Д.С. Оценка алгоритмов машинного обучения. Дилемма смещения-дисперсии / Тишковский М.А., Лимонтов А.С., Подвальников Д.С. // 55-я юбилейная научная конференция аспирантов, магистрантов и студентов. - 2019. - c. 221-223.

\newpage
