\newpage
\begin{center}
\textbf{
\MakeUppercase{Приложение B}\\
\logiql синтаксис}
\end{center}
\addcontentsline{toc}{section}{Приложение B. \logiql синтаксис}

Предикаты:

\begin{lstlisting}[language=Prolog]
sales_u(product, store, day, unit) -> string(product), string(store), datetime(day), int(unit).

returns_u(product, store, day, units) -> string(product), string(store), datetime(day), int(units).

multivalued_predicate(key; value1, value2) -> string(key), int(value1), int(value2).
\end{lstlisting}

Правила:

\begin{lstlisting}[language=Prolog]
netsales_u(product, store, day, net) <-
  sales_u(product, store, day, sls),
  returns_u(product, store, day, returns),
  net = sls - returns.

profit[x] = value <- revenue[x] - cost[x] = value.

+cost(x) <- +new_promo(x).

_(product , tunits) <-
  agg << tunits=total(untis) >>
    sales_u(product , _, _, units).


\end{lstlisting}

Ограничения:

\begin{lstlisting}
sales_u[product, store, day] = unit -> product_catalog(product).

returns_u[product, store, day] = ret,
sales_u[product, store, day] = sls ->
  ret <= sls.
\end{lstlisting}

Пример раскраски графа:

\begin{lstlisting}
node(n), node_id(n:uuid) -> string(uuid).
color(c), color(c:color_name) -> string(color_name).
edge(u, v) -> node(u), node(v).
color_of[n] = c -> node(n), color(c).
node(n) -> color_of[n] = _.
edge(x, y), color_of[x] = c, color_of[y] = c -> x = y.
\end{lstlisting}

Триггеры:

\begin{lstlisting}
^sales_u[product, store  day] = new_units <-
  +sales_line[_, product, store, day] = line_untis,
  sales_u@prev[product, store, day] = units,
  new_units = line_units + units.
\end{lstlisting}
