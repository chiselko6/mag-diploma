\subsection{Тестирование}
\label{sec:technology:testing}

В \LB предусмотрена своя платформа для юнит-тестов - \lstinline{lb unit} \cite{lb_testing}. \lstinline{lb unit} дает разработчикам контроль над транзакциями, обработке данных и тп. Пакеты тестов объединяют файлы с расширением \lstinline{.lb}. Как и в других известных библиотеках, здесь предусмотрены понятия \lstinline{setUp} и \lstinline{tearDown}, которые вызываются соответственно до и после выполнения отдельного теста - они описываются в файлах \lstinline{setUp.lb} и \lstinline{tearDown.lb}.

Использование таких тестов для бенчмарков не совсем оправдано. На то существует несколько причин:

\begin{enumerate}
  \item сложно тестировать действительно долгую обработку данных, поскольку, как правило, это требует больших объемов информации;
  \item такой подход скорее направлен на выявление логических ошибок при разработке, чем на оптимизацию всего приложения.
\end{enumerate}

В связи с этим для тестирования производительности был выбран другой подход, который получил название \emph{\lamias}. Новый подход по сути является некоторой разновидностью нагрузочного тестирования, когда для большого объема данных оценивается скорость работы запросов. Помимо этого (что не касается оптимизации, но также полезно применять) тестируется консистентность данных. Это говорит о том, что при любых изменениях и новых дополнениях те тестовые данные, которые имеются, не должны потерять свою смысловую нагрузку. Например, если в логике подсчета прибыли учесть новый фактор, который снизит прибыль, следовательно, это и должно быть ожидаемо реально. Но если же были внесены некоторые изменения, которые логически не влияли на подсчет прибыли, но по факту значения изменились, то это вызывает вопросы к реализованной логике и требует пересмотреть заново внесенные правки.

\lamias - это консольное приложение, у которого по сути лишь 2 команды:

\begin{enumerate}
  \item \lstinline{make check} – запуск проверки тестов. При выполнении этой команды посылаются описанные в тестовых файлах запросы, которые получают данные от сервера. Это могут быть как и простые запросы значений, так и сложные запросы агрегации. Далее каждый запрос сравнивается с эталонным результатом этого запроса, и, если существует различие в каких-либо значениях, данные запрос и ожидаемые и фактические результаты выводятся на экран. Кроме того, измеряется \emph{время выполнения запроса}.
  \item \lstinline{make update} (не относится к оптимизации) - результатом этой команды будет замена существующих эталонных данных на те, которые получились в результате выполнения команды \lstinline{make check},то есть здесь предполагается, что новые результаты теперь отражают правильную логику работы приложения.
\end{enumerate}

Тестовые пакеты пишутся на языке \js. Пример описания теста \lstinline{resultByProduct}, в котором выполняется очень большой по объему возвращаемых данных, представлен ниже:

\begin{lstlisting}[language=Javascript]
var config = require('./config');
lamias
    .add(Suite('resultByProduct')
        .after(config.clear_adjustments())
        .level('Location', 'store')
        .level('Product', 'item')
        .filter('startWeek', '20180102')
        .filter('endWeek', '20190104')
        .add(Relax('getResultsByProduct')
)
.url('/query/resultByProduct'))
\end{lstlisting}
