\subsection{\acid свойства базы}
\label{sec:technology:acid}

Как и в другим транзакционных системах, база данных \LB отвечает требованиям \acid \cite{lb_acid}.

\begin{enumerate}
  \item \emph{Atomicity} (атомарность). Транзакции поддерживают принцип "все или ничего". Изменения в одной транзакции либо применяются и видны следующим (или параллельным) транзакциям, либо отменяются полностью. В частности, если нарушается какое-либо ограничение, то текущие внесенные изменения не видны другим транзакциям или другим пользователям. Другими словами, в любое момент времени состояние базы описывается последней успешной транзакцией.
  \item \emph{Consistency} (согласованность). Достигается широким набором правил и ограничений в языке \logiql, описывающих отношение данных между предикатами. Правила включают в себя обычные \idb (вычисляемые) правила (например, \lstinline{profit[x] = value <- revenue[x] - cost[x] = value}), а также \edb (явные) (например, \lstinline{+cost(x) <- +new_promo(x).}). Ограничения включают в себя обычные функциональные ограничения (\lstinline{f[x] = y} означает, что для одного значения \lstinline{x} существует не более одного значения \lstinline{y}), а также "правострелочные" правила (\lstinline{cost[x] = value -> value > 0}). Система гарантирует, что коммит изменений совершается лишь после того, как все существующие ограничения выполнены и все правила применены, что говорит о том, что после каждой транзакции состояние базы данных находится в согласованном состоянии. Стоит отметить, что транзакция может быть отклонена \emph{только} в случае \emph{integrity constraints} (т.е. тех, которые описаны здесь). Уникальность конкурентной модели \LB заключается в том, что в отличие от традиционных оптимистичных моделей параллелизма, здесь транзакции не могут быть отклонены по причине параллельного исполнения.
  \item \emph{Isolation} (изолированность). \LB база данных гарантирует наивысший уровень изолированности - \emph{serializable}. На текущий момент транзакции на чтение выполняются параллельно, в то время как запросы на запись сериализуются и выполняются друг за другом. В скором времени планируется добавить поддержку выполнения параллельных транзакций на запись с сохранением \emph{serializable} уровня изолированности.
  \item \emph{Durability} (долговечность). Транзакции имеют дополнительные настройки: \lstinline{softcommit} и \lstinline{diskcommit}. \lstinline{diskcommit}, как видно из названия, возвращает успешный код после того, как данные сохранены на диск. \lstinline{softcommit}, в свою очередь, уведомляет об успешном сохранении в представление в памяти. Таким образом, в случае какого-либо сбоя транзакции с успешным \lstinline{softcommit} могут быть утеряны, в то время как в режиме  \lstinline{diskcommit} гарантированно сохранит результат.
\end{enumerate}

Кроме того, платформа \LB устойчива к \emph{частичным записям} \cite{partial_writes_def}, поэтому данные не могут быть повреждены, если операция записи была остановлена в некоторый момент времени.
