\subsection{\hydra}
\label{sec:technology:hydra}

В настоящий момент существует множество инструментов для поддержки непрерывной интеграции – процесса автоматической и постоянной сборки проекта. Но, как бы то ни было, у всех них нет хорошего варьирования в среде сборки: компиляторы, библиотеки, тестирующие инструменты должны быть уставновлены вручную на всех машинах, на которых устанавливаются автоматические сборки \cite{hydra_description}.

\hydra разработана на основании \nix, является менеджером пакетов, в которой используется функциональный язык для описания установки пакетов и их зависимостей. Это позволяет среде сборки работать автоматически и предсказуемо, что также облегчает обслуживание среды \emph{continuous integration}.

Разберемся в общем принципе работы системы \emph{continuous integration}. Эта система периодически или постоянно следит за изменениями в исходном коде. Каждый раз она собирает проект, тестирует его и генерирует отчет для разработчиков. Таким образом, различные ошибки, которые случайно попали в код, автоматически отлавливаются. Исходя из этого, система может автоматически решать такие вопросы, как:

\begin{itemize}
  \item тестирование переносимости: программы должны быть установлены и протестированы на различных устройствах и платформах. При каждом коммите разработчика это не приходится делать вручную;
  \item процесс тестирования на проектах может занимать целые часы или дни;
  \item статический и динамический анализ;
  \item можно собирать и запускать различные варианты приложений. К примеру, чтобы узнать совместимость с разными версиями компиляторов;
  \item на больших проектах, разработчики разделены на конкретные компоненты приложения, что не позволяет им оценивать полученные взаимодействие с остальными компонентами;
  \item программа может быть автоматически выпущена вместе с необходимыми пакетами, чтобы пользователи могли скачать ее и установить.
\end{itemize}

В самой простой формулировке термин \emph{continuous integration} работает
как бесконечный цикл сборки и выпуска компонентов программного обеспечения системы.

Менеджер пакетов \nix \cite{hydra_manager} имеет исключительные свойства, необходимые для адресации проблем сборки и поддержки вариативности. В основе \nix лежит ленивый чисто функциональный язык (также именуемый \emph{Nix expression language}), который определяет порядок сборки пакетов и их выполнения. Это позволяет собирающей среде быть самовыражаемой, а также представляет изменяемость как набор выполняющихся функций. Ленивость избавляет от такой проблемы, как выполнение ненужных параметров функции.

\nix хранит пакеты таким образом, что они между собой не пересекаются (не перекрывают), что может привести к незапланированным необъявленным зависимостям. Результат сборки каждого пакета хранится в файловой системе под криптографическим хешем всех входных параметров, нужных данному пакету. Например, \linebreak\lstinline{/nix/store/fhmqjrs5sj7b477f8wnlf2588s2jdccb-firefox-3.0.4/}.

Пример выполнения скрипта с помощью \nix:

\begin{lstlisting}[language=Nix]
rec {
  bison = stdenv.mkDerivation {
    name = "bison-2.4";
    src = fetchurl {
      url = mirror://gnu/bison/bison-2.4.tar.bz2;
      sha256= "0c9sv03wsqnqc7wfpa51yc9yy1i3kdgsrjg7qchx0sk8zr11cv qf";
    };
    inherit gnum4;
    buildCommand = ’’
    tar xf $src
    cd bison-*
    export PATH=$gnum4/bin:$PATH
    ./configure --prefix=$out
    make
    make install
    ’’; };
  gnum4 = stdenv.mkDerivation {
    name = "gnum4-1.4.9";
    ...
  };
  stdenv = {
    mkDerivation = args: derivation { ... gcc ... };
  };
  gcc = derivation { ... };
}
\end{lstlisting}
