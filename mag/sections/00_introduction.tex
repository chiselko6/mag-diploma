% \sectioncentered*{Определения и сокращения}
% \label{sec:definitions}
%
% В настоящей пояснительной записке применяются следующие определения и сокращения.
% \\
%
% \emph{Спецификация} -- документ, который желательно полно, точно и верифицируемо определяет требования, дизайн, поведение или другие характеристики компонента или системы, и, часто, инструкции для контроля выполнения этих требований \cite{istqb_specification}.
%
% \emph{Веб-приложение} -- клиент-серверное приложение, в котором клиентом выступает браузер, а сервером -- веб-сервер.
%
% \emph{Кроссплатформенность} -- способность программного обеспечения работать более чем на одной аппаратной платформе и (или) операционной системе.
%
% \emph{Нативное программное средство} -- программное средство, специфичное для какой-либо платформы \cite{habr_crossplatform}.
%
% \emph{Проприетарное программное обеспечение} -- программное обеспечение, являющееся частной собственностью авторов или правообладателей и не удовлетворяющее критериям свободного ПО: свобода запуска программы в любых целях, свобода адаптации программы для любых нужд, свобода распространения, свобода улучшений исходных кодов и публикации улучшений~\cite{free_software}.
% \\
%
% ВУЗ -- высшее учебное заведение.
%
% ПС -- программное средство.
%
% ПО -- программное обеспечение.
%
% БД -- база данных.
%
% СУБД -- система управления базами данных.
%
% ЯП -- язык программирования.
%
% API -- application programming interface (сетевой программный интерфейс).
%
% UI -- user interface (пользовательский интерфейс).
%
% ТЭО -- технико-экономическое обоснование.


\sectioncentered*{Введение}
\addcontentsline{toc}{section}{Введение}
\label{sec:introduction}

Как говорил Сэм Уолтон, главный основатель компании-ритейлера Wal-Mart \cite{retail_management_review}, "после моей карьеры в сфере розничной торговли я застрял на одном направляющем меня принципе: дай потребителю то, что он хочет... Но потребители хотят все: полный ассортимент товара хорошего качества, минимально возможная цена, гарантированное удовлетворение своей покупки, дружественный и осведомленный сервис, удобные часы, свободная парковка и положительный опыт самой покупки. Вам нравится, когда вы посещаете магазин, который каким-то образом превышает ваши ожидания, и вы очень расстраиваетесь, когда магазин разочаровывает вас, или притворяется, что вы невидимы..."

Розничная торговля, или ритейл (от англ. retail – "розничный", "в розницу") – продажа товаров или услуг небольшим количеством, поштучно. Осуществляется через предприятия розничной торговли. Объектом розничной торговли является покупатель, приобретающий товар, предназначенный для личного, семейного, домашнего или иного пользования, не связанного с предпринимательской деятельностью. Субъектом розничной торговли является продавец \cite{retail_definition}.

В нынешнее время продавец заинтересован в удовлетворении потребительских потребностей. Исходя из быстрого и занятого стиля жизни теперешних продавцов, они также предлагают некоторые услуги, в отличие от самих продуктов. Розничная торговля охватывает важные места в экономике страны. Это финальная стадия распределения продуктов либо услуг. Она не только повышает ВВП государства, но также и позволяет большему количеству людей приобрести работу.

Любая организация, которая продает товары для пользования в личных, семейных, бытовых целях потребителей завязана на сфере розничной торговли. При этом редко встречаются такие организации, которые ограничиваются продажей лишь одного продукта. Как правило, выход на рынок связан с целым ассортиментом товаров, в целом имеющие схожие параметры, но принадлежащие к разным категориям жизни деятельности.

В повседневной жизни крупным организациям становится сложнее следить за движением их продуктов. Им постоянно приходится отвечать на следующие вопросы:
\begin{itemize}
  \item где найти товар;
  \item по какой цене купить товар;
  \item сколько товара купить;
  \item куда его доставить;
  \item как его доставить;
  \item по какой цене продать товар.
\end{itemize}

Вместе с высоким темпом роста объемов такой информации появилась необходимость создавать специализированные системы, которые будут не только учитывать все изменения в этих цепочках, но также и автоматически предпринимать действия (отправлять запросы на закупку, генерировать отчеты по продажам). Кроме того, возрастает необходимость анализа совершенных действий. В частности, требуется следить за ходом продаж, чтобы научиться предсказывать их в ближайшем будущем. Также полученная система должна быть достаточно распределенной, для того чтобы управлять ею из разных точек. Дополнительным требованием будет и время обработки результатов – подсчет продаж, агрегация значений \cite{retail_software}.
