\newpage
\begin{flushright}
\textbf{\MakeUppercase{Приложение C}}
\end{flushright}
\begin{center}
\textbf{\MakeUppercase{Примеры оптимизации}}
\end{center}
\addcontentsline{toc}{section}{Приложение C. Примеры оптимизации}

Переупорядочивание ключей:

\begin{lstlisting}[language=Prolog]
fcst_in_range[sku, store, week] = f <-
  fcst[sku, store, week]=f,
  future_week(week).

fcst$2_0_1_3[week, sku, store] = f <- fcst[sku, store, week] = f.

fcst_in_range[sku, store, week] = f <-
  fcst$2_0_1_3[week, sku, store] = f,
  future_week(week).
\end{lstlisting}

Фиксирование порядка ключей:

\begin{lstlisting}[language=Prolog]
fcst_in_range[sku, store, week] = f <-
  pragma_force_key_ordering(sku, store, week),
  fcst[sku, store, week]=f,
  filter(sku, store),
  future_week(week).

binds_pred["pragma_force_key_ordering"]="SKU,STORE, WEEK".
\end{lstlisting}

Переформулировка правил (как должно быть):

\begin{lstlisting}[language=Prolog]
filtered_fcst[sku, store, week] = f <-
  fcst[sku, store, week]=f,
  filter(sku, store),
  active(sku).
\end{lstlisting}

Переформулировка правил (как плохо делать):

\begin{lstlisting}[language=Prolog]
filtered_fcst[sku, store, week] = f <-
  fcst[sku, store, week] = f,
  filter(sku, store),
  future_week(week).
\end{lstlisting}

Переформулировка правил (пример). Было:

\begin{lstlisting}[language=Prolog]
short_sku_alert(promo, sku) <-
  contains(promo, spot),
  promotes(spot, sku),
  starts_on[promo]=day,
  projected_iventory_of[sku, day] < stock_threshold[].
\end{lstlisting}

Стало:

\begin{lstlisting}[language=Prolog]
skus_short_at_start(promo, sku) <-
  starts_on[promo]=day,
  shortages(sku, day).

short_sku_alert(promo, sku) <-
  skus_on_promo(promo, sku),
  skus_short_at_start(promo, sku).

skus_on_promo(promo, sku) <-
  contains(promo, spot),
  promotes(spot, sku).

shortages(sku, day) <-
  projected_iventory_of[sku, day] < stock_threshold[].
\end{lstlisting}
