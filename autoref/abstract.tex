\sectioncentered*{Общая характеристика работы}
\addcontentsline{toc}{section}{Общая характеристика работы}

\textbf{Цель и задачи исследования}

\emph{Целью} диссертационной работы является  оптимизация программного продукта на платформе \LB, автоматизирующего операции с данными в сфере розничной торговли, для увеличения производительности обработки сложных запросов и быстродействия при пересчете планов.

\emph{Объектом} исследования является готовый программный продукт, используемый крупными торговыми компаниями.

\emph{Предметом} исследования является исходный код продукта, платформа и база данных \LB.\\


% \textbf{Связь работы с приоритетными направлениями научных исследова- ний и запросами реального сектора экономики}

\textbf{Личный вклад соискателя}

Результаты, приведенные в диссертации, получены соискателем лично при изучении и анализе работы платформы \LB.\\

% \textbf{Апробация результатов диссертации}

\textbf{Опубликованность результатов диссертации}

Опубликовано 2 тезиса в сборнике трудов и материалов конференций.\\

\textbf{Структура и объем диссертации}

% \thispagestyle{empty}
Диссертация состоит из введения, общей характеристики работы, трех глав, заключения, списка использованных источников, списка публикаций автора. В первой главе произведен обзор предметной области - какие требования ставятся при построении приложений для розничной торговли, как разрабатываются такие приложения в общих случаях и какие частные практики используются при построении на платформе \LB. Вторая глава посвящена используемым технологиям, в частности дается разбор составляющих платформы, используемого языка запросов, веб-сервисов. Третья глава содержит непосредственно детальный разбор проблем при работе с базой данных и их решений. В заключении подводится краткий итог всех примененных оптимизаций.

% Зачем: чтобы можно было вывести общее число страниц.
% Добавляется единица, поскольку последняя страница -- ведомость.
\FPeval{\totalpages}{round(\getpagerefnumber{LastPage} + 1, 0)}

Общий объем работы составляет 73 с., 21 рис., 1 табл., 1 формула, 31 источник.
