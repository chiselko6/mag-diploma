\sectioncentered*{Краткое введение}
\setcounter{page}{2}

Розничная торговля, или ритейл (от англ. retail – "розничный", "в розницу") – продажа товаров или услуг небольшим количеством, поштучно. Осуществляется через предприятия розничной торговли. Объектом розничной торговли является покупатель, приобретающий товар, предназначенный для личного, семейного, домашнего или иного пользования, не связанного с предпринимательской деятельностью. Субъектом розничной торговли является продавец.

В нынешнее время продавец заинтересован в удовлетворении потребительских потребностей. Исходя из быстрого и занятого стиля жизни теперешних продавцов, они также предлагают некоторые услуги, в отличие от самих продуктов. Розничная торговля охватывает важные места в экономике страны. Это финальная стадия распределения продуктов либо услуг. Она не только повышает ВВП государства, но также и позволяет большему количеству людей приобрести работу.

Любая организация, которая продает товары для пользования в личных, семейных, бытовых целях потребителей завязана на сфере розничной торговли. При этом редко встречаются такие организации, которые ограничиваются продажей лишь одного продукта. Как правило, выход на рынок связан с целым ассортиментом товаров, в целом имеющие схожие параметры, но принадлежащие к разным категориям жизни деятельности.

В повседневной жизни крупным организациям становится сложнее следить за движением их продуктов. Им постоянно приходится отвечать на следующие вопросы:
\begin{itemize}
  \item где найти товар;
  \item по какой цене купить товар;
  \item сколько товара купить;
  \item куда его доставить;
  \item как его доставить;
  \item по какой цене продать товар.
\end{itemize}

Вместе с высоким темпом роста объемов такой информации появилась необходимость создавать специализированные системы, которые будут не только учитывать все изменения в этих цепочках, но также и автоматически предпринимать действия (отправлять запросы на закупку, генерировать отчеты по продажам). Кроме того, возрастает необходимость анализа совершенных действий. В частности, требуется следить за ходом продаж, чтобы научиться предсказывать их в ближайшем будущем. Также полученная система должна быть достаточно распределенной, для того чтобы управлять ею из разных точек. Дополнительным требованием будет и время обработки результатов – подсчет продаж, агрегация значений.

Чтобы глубже понять саму проблему и перенести ее на математический язык, стоит построить упрощенные модели. Типичными моделями здесь будут являться:
\begin{itemize}
  \item финансовые модели ​− сколько продавец хочет выручить в ближайшее время при определенных условиях;
  \item модели данных – это могут быть скидки, распродажи, акции;
  \item модели сети цепей поставок – процесс доставки до конечного пользователя.
\end{itemize}

Но, даже построив приведенные модели, сложно понять, как именно проходят процессы в этой среде. Многие продавцы создают информативные таблицы. Головную боль здесь доставить могут не столько большие таблицы, сколько много маленьких, у которых зачастую еще и разный формат содержания. Количество таких таблиц может достигать нескольких тысяч. Отсюда появляется сложность перевода этих данных в необходимые математические модели. Самые ценные модели – такие модели, которые могут быть построены и поддерживаемы экспертами, при этом их можно без особых трудностей изменять во времени.

Стоит сфокусироваться на тех моделях, которые зачастую используют продавцы. Пожалуй, самая простая, какую можно представить, будет следующая формула \lstinline{Profit = revenue - cost}.

Основная задачи этой модели ​− максимизация функции \emph{profit}, которая зависит от двух параметров: \emph{revenue} (выручка) и \emph{cost} (затраты).
